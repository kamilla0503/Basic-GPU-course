\documentclass{article}
\usepackage[utf8]{inputenc}
\usepackage{graphicx}
\graphicspath{ {./images/} }
\usepackage{amsmath}
\usepackage{hyperref}
%\usepackage[english]{babel}

\usepackage[left=2cm,right=1cm, top=2cm,bottom=2cm,bindingoffset=0cm]{geometry}

\renewcommand{\normalsize}{\fontsize{14}{18pt}\selectfont}

\title{A TensorFlow example}
\author{ Kamilla Faizullina}
\date{\empty}

\begin{document}


\maketitle
  We implement Conway’s Game of Life using TensorFlow and run the program either using GPU or CPU.
 

\section{Implementation}
 For the main function we use built-in conv2d function which applies a 2D convolution over an  state of the simulation. 
 To count information from neighbors cells, we can use kernel consisting of ones. 
 \begin{verbatim}
  kernel = tf.reshape(tf.ones([3,3], dtype=board.dtype), [3,3,1,1])
 for i in range(iters):      
 	neighbours = tf.nn.conv2d(board, kernel, [1,1,1,1], "SAME") - board
 	survive = tf.logical_and(tf.equal(board, 1), tf.equal(neighbours, 2))
 	born = tf.equal(neighbours, 3)
 	board = tf.cast(tf.logical_or(survive, born), board.dtype)
 \end{verbatim}
 \begin{figure}[hp]
 	\includegraphics[scale=0.17]{start.png} 
 	\includegraphics[scale=0.17]{end.png}  \\
 	\includegraphics[scale=0.97]{test.png} 
 	\caption{ First and Last states. I made tests on Google Collab to validate the program.  }
 	\label{states}
 \end{figure}






\section{ Performance for CPU and the GPU }
 Tensorflow allows to run program on the GPU (I suppose it is quite optimized in comparison to my own implementations).
 
\begin{table}[htp]
	\centering
	\begin{tabular}{||c c c   ||} 
		\hline
		Type  & GPU time  &  CPU time   \\ [0.5ex] 
		\hline\hline
		 tf.float16 & 12.461 & 49.87 \\ 
		tf.float32 &   11.22 & 40.872 \\ 
		tf.float64 & 12.685 & 44.837 \\
		 tf.int32 & 19.8 &50.715 \\	[1ex] 
		\hline
	\end{tabular}
	\caption{Results }
	\label{table:2}
\end{table}

Table \ref{table:2} presents obtained run time for four different data types. In all cases, GPU had more performance.  
 
%\includegraphics[scale=0.6]{fow_prev_trim.png}
 

 

%\end{thebibliography}

\end{document}
